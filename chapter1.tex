
% chapter1.tex
\chapter{Functions of One Complex Variable}

\section*{Introduction}

Our aim in this chapter is to develop the familiar theories of analytic functions of one complex variable and Riemann surfaces in a way that generalises well to the several variable theory. In §3 we see how the existence theory of the Cauchy-Riemann equations can be used to prove the Mittag-Leffler theorem and also how the topology of domains in \( \mathbb{C} \) naturally enters into the proof of the Weierstrass theorem. In §§4,5 we show how the theory of holomorphic line bundles may be used to reformulate some of the classical problems in Riemann surface theory. We also define the Cauchy-Riemann equations on an arbitrary Riemann surface and indicate how they are related to the problem of constructing meromorphic functions with specified divisors. In an appendix we prove a number of classical results, including the Runge approximation theorem. We use the Runge theorem to construct solutions of the Cauchy-Riemann equations.

\section{Analytic functions and power series}

Let \( \Omega \) be a domain in \( \mathbb{C} \). We recall that a function \( f: \Omega \to \mathbb{C} \) is said to be *analytic* or *holomorphic* if it is complex differentiable on \( \Omega \). Writing \( f \) in real and imaginary parts, \( f = u + iv \), analyticity implies that \( u \) and \( v \) satisfy the Cauchy-Riemann equations on \( \Omega \):

\[\frac{\partial u}{\partial x} = \frac{\partial v}{\partial y}; \quad \frac{\partial u}{\partial y} = -\frac{\partial v}{\partial x}.\]

Recalling that a real \( 2 \times 2 \)-matrix \( [a_{ij}] \) induces a complex linear endomorphism of \( \mathbb{C} \) if and only if \( a_{11} = a_{22} \) and \( a_{12} = -a_{21} \), we may interpret the Cauchy-Riemann equations as saying that if \( f \) is analytic then \( f \) is differentiable in the real sense and the (real) derivative of \( f \) is everywhere a complex linear map (see, for example, Field [1; Example 3, page 133]).

Let \( A(\Omega) \) denote the set of all analytic functions on \( \Omega \).

We now introduce a pair of partial differential operators which, together with their generalisations to several variables, will be of the utmost importance in the sequel. We set

\[
\frac{\partial}{\partial z} = \frac{1}{2}\left(\frac{\partial}{\partial x} - i\frac{\partial}{\partial y}\right); \quad \frac{\partial}{\partial \bar{z}} = \frac{1}{2}\left(\frac{\partial}{\partial x} + i\frac{\partial}{\partial y}\right).
\]

For \( r \geq 1 \),

\[
\frac{\partial}{\partial z}, \frac{\partial}{\partial \bar{z}}: C^r(\Omega) \rightarrow C^{r-1}(\Omega).
\]

The significance of these operators may be gauged from

\begin{lemma}
A function \( f \in C^1(\Omega) \) is analytic if and only if \( \partial f / \partial \bar{z} = 0 \).
\end{lemma}

\begin{proof}
The reader may verify that \( \partial f / \partial \bar{z} = 0 \) iff the Cauchy-Riemann equations hold. Since \( f \) is assumed to be \( C^1 \), the Cauchy-Riemann equations hold iff \( f \) is analytic.
\end{proof}

Next we recall the basic theorem on the local representation of analytic functions by power series.

\begin{theorem}
Let \( f \in A(\Omega) \). Given \( \zeta \in \Omega \) and \( r > 0 \) such that \( D_r(\zeta) \subset \Omega \), we have

\[
f(z) = \sum_{j=0}^{\infty} a_j(z - \zeta)^j, \quad z \in D_r(\zeta),
\]

where \( a_j = \partial^j f / \partial z^j (\zeta) / j! \), and convergence is uniform on compact subsets of \( D_r(\zeta) \).
\end{theorem}

\begin{remark}
Simple examples, such as \( f(z) = (1 - z)^{-1}, \Omega = \mathbb{C} \setminus \{1\} \), show that the power series at \( \zeta \) need not converge on the whole of \( \Omega \).
\end{remark}

\begin{corollary}
An analytic function is \( C^\infty \).
\end{corollary}

\begin{remark}
Notice that \( A(\Omega) = \text{Kernel}(\partial / \partial \bar{z}) \). Now \( \partial / \partial \bar{z} \) is an example of an elliptic differential operator and it can be shown that the kernel of any elliptic operator consists of \( C^\infty \) functions. We shall return to this type of question in later chapters.
\end{remark}

\begin{corollary}
If \( f \in A(\Omega) \), then \( \partial^j f / \partial z^j \in A(\Omega) \), \( j \geq 0 \).
\end{corollary}

\begin{corollary}
Let \( \Omega \) be a domain in \( \mathbb{C} \). Suppose \( f \in A(\Omega) \) and that at some point \( \zeta \in \Omega \), \( \partial^j f / \partial z^j (\zeta) = 0 \), \( j \geq 0 \). Then \( f \) vanishes identically on \( \Omega \).
\end{corollary}

\begin{proof}
Let \( X = \{z \in \Omega: \partial^j f / \partial z^j = 0, \, \text{all } j \geq 0\} \). \( X \) is open by the power series representation of analytic functions given by Theorem 1.1.2 and \( X \) is certainly non-empty since \( \zeta \in X \). Since \( X \) is the intersection of the closed sets \( \{z \in \Omega: \partial^j f / \partial z^j (z) = 0\} \), \( X \) is also closed. Since \( \Omega \) is connected, \( X = \Omega \).
\end{proof}

\begin{remark}
Another way of stating this corollary is that the value of an analytic function and all its derivatives at a single point of a domain determine the function uniquely. This type of behaviour does not, of course, hold for \( C^\infty \) or continuous functions.
\end{remark}

\begin{proposition}[Uniqueness of analytic continuation]
Let \( U, V \) be connected open subsets of \( \mathbb{C} \) and suppose \( U \cap V \neq \emptyset \). If \( f \in A(U) \) and \( h \) is an analytic extension of \( f \) to \( U \cup V \) (that is, \( h \in A(U \cup V) \) and \( h | U = f \)), then \( h \) is unique.
\end{proposition}

\begin{proof}
If \( h_1, h_2 \) are analytic extensions of \( f \) to \( U \cup V \) then \( h_1 - h_2 \) is an analytic extension of the zero function on \( U \) to \( U \cup V \). Therefore, \( h_1 - h_2 \) is identically zero by Corollary 1.1.5.
\end{proof}

\begin{remark}
Once we have uniqueness of analytic continuation it is natural to try to construct the largest domain to which any given analytic function may be extended. It turns out of course that we have to enlarge our class of domains to include Riemann surfaces spread over \( \mathbb{C} \). We return to this question in §4 of this chapter and again in §2 of Chapter 6.
\end{remark}

\begin{xcb}{Exercises}
These exercises are revision of basic theory of functions of one complex variable. Proofs may be found in any of the many introductory texts on complex analysis.

\begin{enumerate}
\item (Laurent series). Let \( f \) be analytic on the annulus \( r < |z - z_0| < R \). Derive the Laurent series of \( f \) at \( z_0 \),

\[
f(z) = \sum_{n=-\infty}^{+\infty} a_n (z - z_0)^n, \quad r < |z - z_0| < R,
\]

where \( a_n = (2\pi i)^{-1} \int_{|\zeta - z_0| = s} f(\zeta)/(\zeta - z_0)^{n+1} d\zeta, \quad r < s < R \), and convergence is uniform on compact subsets of the annulus.

\item (Cauchy's inequalities). Continuing with the notation and assumptions of question 1, show that if \( M(t) = \sup\{|f(z)|: |z - z_0| = t\}, r < t < R \), then
\[
|a_n| \leq M(t)/t^n, \quad n \in \mathbb{Z}.
\]

In particular, show that if f is holomorphic on the disc \( D_R(z_0) \) then
\[
|\partial^n f / \partial z^n (z_0)| \leq M(t)n!/t^n, \quad n \geq 0.
\]

\item (Riemann removable singularities theorem). Suppose that f is an analytic function in the punctured disc \( D_r(z_0)^* = \{z: 0 < |z - z_0| < r\} \). Show that a necessary and sufficient condition for f to extend analytically to \( D_r(z_0) \) is that f is bounded on some neighbourhood of \( z_0 \). (Use the result of question 2).

\item (Open mapping theorem). Let f be analytic and not identically zero on the domain U in \( \mathbb{C} \). Prove that \( f(U) \) is open in \( \mathbb{C} \).

\item (Monodromy theorem). Let \( \Omega \) be a domain in \( \mathbb{C} \), \( z_0, y_0 \in \Omega \) and suppose f is analytic on some neighbourhood U of \( z_0 \). Let \( C \) be a continuous path in \( \Omega \) parametrized by \( \phi: [0,1] \to \Omega \) with \( \phi(0) = z_0 \), \( \phi(1) = y_0 \). We say f can be analytically continued along C if we can find discs \( D_i = D_{r_i}(\phi(t_i)), 0 = t_0 < \ldots < t_k = 1 \), covering \( C \), \( h_i \in A(D_i) \), i = 0, \ldots, k, such that \( h_0 = f \) on U \(\cap\) \( D_0 \) and \( h_i = h_{i+1} \) on \( D_i \cap D_{i+1} \), i \(\geq\) 0. Define \( f_C(y_0) \) to be \( h_K(y_0) \). Show that

\begin{enumerate}
\item \( f_C(y_0) \) depends only on C and not on any of the choices we have made.
\item If \( C, C' \) are homotopic curves in \( \Omega \) joining \( z_0 \) to \( y_0 \) then
\[
f_C(y_0) = f_{C'}(y_0).
\]
\end{enumerate}

Give examples to show that if \( \Omega \) is not simply connected and \( C, C' \) are non-homotopic curves joining \( z_0 \) to \( y_0 \) then \( f_C(y_0) \) may not equal \( f_{C'}(y_0) \).

\item (Maximum principle). Let f be analytic on the domain \( \Omega \) in \( \mathbb{C} \). If \( |f| \) has a maximum in \( \Omega \) then f is constant.

\item (Schwarz' lemma). Suppose \( f \in A(D_1(0)) \) and satisfies \( f(0) = 0 \) and \( |f(z)| \leq 1 \), \( z \in D_1(0) \). Then \( |f(z)| \leq |z| \) and \( |f'(0)| \leq 1 \). Equality holds if and only if \( f(z) = cz \), where \( |c| = 1 \) (Hint: Apply the maximum principle to the function \( f(z)/z \)).
\end{enumerate}
\end{xcb}

\section{Meromorphic functions}

Roughly speaking meromorphic functions are the analytic analogue of rational functions and in this section we briefly review their definition and elementary properties. Throughout the section \( \Omega \) will denote a domain in \( \mathbb{C} \).

Let \( \zeta \in \Omega \) and \( \Omega' = \Omega \setminus \{\zeta\} \). Suppose \( f \in A(\Omega') \). By Laurent's expansion we have
\[
f(z) = \sum_{j=-\infty}^{\infty} a_j(z - \zeta)^{j}, \quad z \in D_{\Gamma}(\zeta)^* \subset \Omega.
\]

There are three possibilities:

\begin{enumerate}
\item[(a)] f is bounded on some neighbourhood of \( \zeta \). In this case \( a_j = 0 \), \( j < 0 \), and f extends uniquely to an analytic function on \( \Omega \) (Riemann removable singularities theorem).

\item[(b)] \( f(z) \to \infty \) as \( z \to \zeta \). Here one can show that there exists a strictly positive integer N such that \( a_j = 0 \), \( j < -N \) and \( a_{-N} \neq 0 \).

\item[(c)] Neither (a) or (b) occurs (\( \zeta \) is an essential singularity of f).
\end{enumerate}

In case (b), f is an example of a meromorphic function on \( \Omega \) with a single pole of order N at \( \zeta \). We may write \( f(z) = u(z)/(z - \zeta)^N \), \( z \in D_{\Gamma}(\zeta)^* \), where \( u \in A(D_{\Gamma}(\zeta)^*) \) and the power series of \( u \) at \( \zeta \) is given explicitly as the Laurent series of f at \( \zeta \) multiplied by \( (z - \zeta)^N \). We may extend \( u \) analytically to \( \Omega \) by taking \( u(z) = (z - \zeta)^N f(z) \), \( z \neq \zeta \). In this way we may represent f as the quotient \( u(z)/(z - \zeta)^N \) of analytic functions defined on all of \( \Omega \).

There are some problems in giving a satisfactory general definition of a meromorphic function. If we attempt to define a meromorphic function on \( \Omega \) as a quotient \( f/g \), \( f,g \in A(\Omega) \), $g \neq 0$, we are faced with the difficulty that f and g may have infinitely many common zeros. If this happens we cannot cancel the zeros using the elementary power series techniques of the previous paragraph to obtain the maximal subdomain of \( \Omega \) on which the meromorphic function is defined as an analytic function. That is, the representation \( f/g \) may be rather non-canonical. We start by giving a definition that is rather special to functions of one complex variable and then show how to reformulate the definition in a way that generalises well to functions of several complex variables.

\begin{definition}
We say that m is a meromorphic function on \( \Omega \) if there exists a discrete subset X of \( \Omega \) such that

\begin{enumerate}
\item m is an analytic function on \( \Omega \setminus X \).
\item Every point of X is a pole of m.
\end{enumerate}
\end{definition}

We notice that the definition excludes essential singularities and so, for example, \( e^{-1/z} \) would not define a meromorphic function on \( \mathbb{C} \).

We denote the set of meromorphic functions on \( \Omega \) by \( M(\Omega) \).

Locally a meromorphic function can be expressed as a quotient of analytic functions. That is, given \( m \in M(\Omega) \) and \( z \in \Omega \), we can find an open neighbourhood U of z and f,g \( \in A(U) \) such that g is not identically zero and \( m|U = f/g \) outside of any poles of m in U. Of course, if U does not contain any poles of m, we can take g \( \equiv \) 1.

We now work towards an alternative definition of a meromorphic function which is framed in terms of local information and requires no explicit information about the pole set.

Suppose that \( \{U_i: i \in I\} \) is an open cover of \( \Omega \) and that for each \( i \in I \) we are given \( f_i, g_i \in A(U_i) \) with \( g_i \) not vanishing identically on any connected component of \( U_i \). Then \( \{(f_i, g_i): i \in I\} \) defines a meromorphic function m on \( \Omega \) provided that for all \( i,j \in I \) we have

\[
f_i/g_i = f_j/g_j
\]

at all points of \( U_i \cap U_j \) where both \( g_i \) and \( g_j \) are non-zero. (Equivalently, \( f_i g_j = f_j g_i \) on \( U_i \cap U_j \)). We omit the routine construction of m. If \( \{V_j : j \in J\} \) is another open cover of \( \Omega \) and \( \{(a_j, b_j) : j \in J\} \) a corresponding set of analytic functions satisfying the above conditions, it is easily verified that \( \{(f_i, g_i) : i \in I\} \) and \( \{(a_j, b_j) : j \in J\} \) define the same meromorphic function if and only if

\[
f_i / g_i = a_j / b_j
\]

at all points of \( U_i \cap V_j \) where both \( g_i \) and \( b_j \) are non-zero.

We can now use this condition to define an equivalence relation on all sets of pairs of analytic functions \( \{(f_i, g_i) : i \in I\} \) satisfying the requisite compatibility conditions. The equivalence classes of this relation are then defined to be meromorphic functions. This is essentially the approach that we adopt in later chapters.

One immediate consequence of our local description of meromorphic functions is that \( M(\Omega) \) is a field (the connectedness of \( \Omega \) is essential here to avoid zero divisors).

Suppose \( m \in M(\Omega) \) and \( \zeta \in \Omega \). We define the order of \( m \) at \( \zeta \), \( \text{ord}(m, \zeta) \), to be the smallest index with non-zero coefficient in the Laurent expansion of \( m \) at \( \zeta \). That is, if

\[
m(z) = \sum_{j=N}^{\infty} a_j (z - \zeta)^j
\]

on some neighbourhood of \( \zeta \) and \( a_N \neq 0 \), then \( \text{ord}(m, \zeta) = N \). Clearly, if \( m = f/g \) in some neighbourhood of \( \zeta \), \( \text{ord}(m, \zeta) = \text{ord}(f, \zeta) - \text{ord}(g, \zeta) \) though of course the terms on the right hand side depend on the choice of local representation for \( m \).

If \( \text{ord}(m, \zeta) > 0 \), we say that \( m \) has a zero of order \( \text{ord}(m, \zeta) \) at \( \zeta \) and if \( \text{ord}(m, \zeta) < 0 \), we say that \( m \) has a pole of order \( -\text{ord}(m, \zeta) \) at \( \zeta \). We set

\[
Z(m) = \{z \in \Omega: \text{ord}(m, z) > 0\} = \{z \in \Omega: m(z) = 0\}
\]

\[
P(m) = \{z \in \Omega: \text{ord}(m, z) < 0\} = \{z \in \Omega: m \text{ has a pole at } z\}
\]

Z(m) and P(m) are called the zero and pole set of \( m \) respectively. Clearly Z(m) and P(m) are disjoint discrete subsets of \( \Omega \) (assuming \( m \neq 0 \)).

We now introduce some useful definitions and notation. Suppose \( p: \Omega \rightarrow \mathbb{Z} \) and \( \{z \in \Omega : p(z) \neq 0\} \) is a discrete subset of \( \Omega \). We call the formal sum \( \sum_{z \in \Omega} p(z) \cdot z \) a divisor on \( \Omega \). We denote the set of divisors on \( \Omega \) by \( D(\Omega) \). \( D(\Omega) \) has the structure of an ordered abelian group if we define
\[
\left( \sum_{z \in \Omega} p(z) \cdot z \right) + \left( \sum_{z \in \Omega} q(z) \cdot z \right) = \sum_{z \in \Omega} (p+q)(z) \cdot z
\]
\[
\sum_{z \in \Omega} p(z) \cdot z > \sum_{z \in \Omega} q(z) \cdot z \quad \text{if and only if}
\]
\( p(z) \geq q(z) \) for all \( z \in \Omega \) with strict inequality for at least one point of \( \Omega \).

Let \( M^*(\Omega) \) denote the group of invertible elements of \( M(\Omega) \). Since \( \Omega \) is assumed connected, \( M^*(\Omega) \) is all of \( M(\Omega) \) except the zero function. Given \( m \in M^*(\Omega) \), the divisor of \( m \), \( \text{div}(m) \), is defined to be
\[
\sum_{z \in \Omega} \text{ord}(m,z) \cdot z
\]
\( \text{div}: M^*(\Omega) \rightarrow D(\Omega) \) is a homomorphism (relative to the multiplicative structure on \( M^*(\Omega) \)). We note that \( \text{div}(m) \geq 0 \) if and only if \( m \in A(\Omega) \).

Suppose \( m \in M(\Omega) \) and \( \zeta \in \Omega \). Let \( m \) have Laurent series
\[
\sum_{j=N}^{\infty} a_j(z-\zeta)^j
\]
at \( \zeta \), where we suppose \( a_N \neq 0 \). The principal part of \( m \) at \( \zeta \) is defined to be
\[
\sum_{j=N}^{-1} a_j(z-\zeta)^j
\]
if \( N \leq -1 \) and zero otherwise. We note that \( m,m' \in M(\Omega) \) have the same principal part at \( \zeta \) if and only if \( m-m' \) is analytic on some neighbourhood of \( \zeta \).

To conclude this section we remark that Proposition 1.1.6 generalises to meromorphic functions and therefore we can discuss meromorphic continuation. We leave details to the reader.

\begin{xcb}{Exercises}
\begin{enumerate}
\item Verify that
\[
\text{div}(mm') = \text{div}(m) + \text{div}(m'), \quad m,m' \in M^*(\Omega).
\]
\[
\text{div}(m^{-1}) = -\text{div}(m), \quad m \in M^*(\Omega).
\]
\[
\text{div}(m) = 0 \text{ if and only if } m \text{ is a nowhere vanishing analytic function on } \Omega.
\]

\item Let \( m \in M^*(\Omega) \). Show that if \( \text{div}(m) = 0 \) then either \( m \) is constant or \( m \) has an essential singularity at infinity.

\item Under what conditions is the composition of two meromorphic functions meromorphic?
\end{enumerate}
\end{xcb}

\section{Theorems of Weierstrass and Mittag-Leffler}

In the preceding sections we have reviewed some of the basic elementary properties of analytic and meromorphic functions. However, we have as yet given no means of constructing such functions so as to satisfy specified properties. For example, if \( X \) is any closed subset of \( \Omega \) it is not difficult to construct a \( C^\infty \) function on \( \Omega \) with zero set \( X \). Can we find an analytic function whose zero set is equal to \( X? \) Clearly we cannot unless \( X \) is a discrete subset of \( \Omega \). It turns out though that if \( X \) is discrete we can always find an analytic function on \( \Omega \) with zero set \( X \). This is exactly the type of result we need if our study of analytic functions is to amount to much more than a study of polynomials, rational functions and the standard analytic functions such as log and exp.

Our aim in this section will be to show the importance of the theory of the partial differential operator \( \partial / \partial \bar{z} \) and the topology of domains in \( \mathbb{C} \) in questions involving the construction of analytic and meromorphic functions with specified behaviour at prescribed poles and zeros. We adopt this approach because it generalises well to functions of several complex variables and complex manifolds. We must emphasise, however, that the Mittag-Leffler and Weierstrass theorems can be given rather more elementary proofs than those presented here which do not depend on the theory of \( \partial / \partial \bar{z} \) (see, for example, Heins [1] or Hille [1]).

Throughout this section \( \Omega \) will denote an open subset of \( \mathbb{C} \).

We give the proof of the following basic existence theorem in the appendix to this chapter.

\begin{theorem}
Let \( f \in C^\infty(\Omega) \). Then there exists \( u \in C^\infty(\Omega) \) such that
\[
\partial u / \partial \bar{z} = f.
\]
\end{theorem}

\begin{remark}
An equivalent formulation of Theorem 1.3.1 is that the sequence
\[
0 \rightarrow A(\Omega) \xrightarrow{i} C^\infty(\Omega) \xrightarrow{\partial / \partial \bar{z}} C^\infty(\Omega) \rightarrow 0
\]
is exact for every open subset \( \Omega \) of \( \mathbb{C} \) (i denotes inclusion).
\end{remark}

The Mittag-Leffler theorem gives conditions under which there exists a meromorphic function on \( \Omega \) with specified principal parts.

Before stating the Mittag-Leffler theorem we introduce some useful notation. Suppose \( U_i \), \( U_j \) are sets, then we let \( U_{ij} \) denote the intersection \( U_i \cap U_j \). We use the obvious generalisation of this notation for intersections of more than two indexed sets.

\begin{theorem}[Mittag-Leffler theorem]
Let \( \{U_i: i \in I\} \) be an open cover of \( \Omega \) and suppose we are given \( m_i \in M(U_i) \) for each \( i \in I \). Then, provided that \( m_i - m_j \in A(U_{ij}) \) for all \( i,j \in I \), there exists \( m \in M(\Omega) \) such that
\[
m - m_i \in A(U_i), \quad i \in I.
\]
\end{theorem}

\begin{remark}
An alternative formulation of this theorem would be: Suppose \( X \) is a discrete subset of \( \Omega \) and that for each \( z \in X \) we are given a meromorphic function \( m^z \) which is defined on some neighbourhood of \( z \) and has a single pole at \( z \). Then there exists \( m \in M(\Omega) \) with pole set \( X \) and such that the principal part of \( m \) at \( z \) equals the principal part of \( m^z \) at \( z \) for all \( z \in X \).
\end{remark}

\begin{proof}[Proof of theorem]
(Following Hörmander [1]). Observe that it suffices to construct \( f_i \in A(U_i) \) such that for all \( i, j \)

\[
m_i + f_i = m_j + f_j \text{ on } U_{ij}.
\]

Indeed, if these compatibility conditions hold we can define \( m \in M(\Omega) \) by setting \( m|U_i = m_i + f_i \) and \( m \) will then obviously satisfy the requirements of the theorem.

What we shall do is to start by constructing \( h_i \in C^\infty(U_i) \) such that \( m_i + h_i = m_j + h_j \) on \( U_{ij} \). This will only use the theory of \( C^\infty \) functions and no complex analysis. Using Theorem 1.3.1, we then construct a "correction" term \( u \in C^\infty(\Omega) \) such that for all \( i \in I \),

\[
h_i + u \in A(U_i).
\]

It is then enough to take \( f_i = h_i + u \).

Step 1. Choose a partition of unity \( \{\theta_\alpha: \alpha \in \Lambda\} \) subordinate to the cover \( \{U_i: i \in I\} \). That is, each \( \theta_\alpha \) is a positive \( C^\infty \) function with compact support, \( \operatorname{supp}(\theta_\alpha) \subseteq U_{\tau(\alpha)} \) for some \( \tau(\alpha) \in I \), only finitely many \( \theta_\alpha \) are non-zero on any given compact subset of \( \Omega \) and \( \sum_{\alpha \in \Lambda} \theta_\alpha \equiv 1 \) on \( \Omega \) (for the construction of partitions of unity we refer to Lang [1], Spivak [1] or de Rham [1]).

Set \( f_{ij} = m_i - m_j \in A(U_{ij}) \) and observe that for all \( i,j,k \in I \)

\[
f_{ij} + f_{jk} + f_{ki} = 0 \text{ on } U_{ijk} \text{ and } f_{ij} = -f_{ji} \text{ on } U_{ij} \quad \ldots (A)
\]

We define

\[
h_i = \sum_{\alpha \in \Lambda} \theta_\alpha f_{\tau(\alpha)i}.
\]

Setting \( f_{\tau(\alpha)i} = 0 \) outside \( U_{\tau(\alpha)i} \) it is clear that \( h_i \in C^\infty(U_i) \). Moreover

\[
h_j - h_i = \sum_{\alpha \in \Lambda} \theta_\alpha (f_{\tau(\alpha)j} - f_{\tau(\alpha)i}) = \sum_{\alpha \in \Lambda} \theta_\alpha f_{ij} \text{ by } (A) = f_{ij} = m_i - m_j.
\]

Hence we have found \( h_i \in C^\infty(U_i) \) satisfying \( m_i + h_i = m_j + h_j \) on \( U_{ij} \).

Step 2. Since \( h_i - h_j \in A(U_{ij}) \),

\[
\partial h_i / \partial \bar{z} = \partial h_j / \partial \bar{z} \text{ on } U_{ij}.
\]

Hence we may define \( F \in C^\infty(\Omega) \) by requiring that \( F|U_i = \partial h_i / \partial \bar{z} \).

To construct \( u \in C^\infty(\Omega) \) satisfying \( u + h_i \in A(U_i) \) for all \( i \in I \), we require \( \partial u / \partial \bar{z} + \partial h_i / \partial \bar{z} = 0 \text{ on } U_i \). That is,

\[
\partial u / \partial \bar{z} = -F \text{ on } \Omega.
\]

By Theorem 1.3.1, there exists \( u \in C^\infty(\Omega) \) satisfying this equation. Finally we take \( f_i = u + h_i \in A(U_i) \) and define \( m \) as indicated in the introduction to the proof.
\end{proof}

The relation (A) occurring in the proof of Theorem 1.3.2 is usually referred to as a cocycle condition and \( \{f_{ij}: i,j \in I\} \) as a cocycle. We shall meet this type of relation frequently in the sequel. The proof of Theorem 1.3.2 clearly proves the slightly stronger

\begin{theorem}
Let \( f_{ij} \in A(U_{ij}) \) satisfy the cocycle conditions

\[
f_{ij} + f_{jk} + f_{ki} = 0 \text{ on } U_{ijk} \text{ and } f_{ij} = -f_{ji} \text{ on } U_{ij}, \text{ all } i,j,k.
\]

Then there exist \( g_i \in A(U_i) \) such that for all \( i,j \),

\[
f_{ij} = g_j - g_i \text{ on } U_{ij}.
\]
\end{theorem}

Notation. If \( U \) is an open subset of \( \mathbb{C} \), we shall let \( A^*(U) \) denote the set of nowhere vanishing analytic functions on \( U \). That is, the units in \( A(U) \).

\begin{theorem}[Weierstrass theorem]
Let \( U = \{U_i: i \in I\} \) be an open cover of \( \Omega \) and suppose that we are given \( m_i \in M^*(U_i) \) for each \( i \in I \). Then, provided that \( m_i / m_j \in A^*(U_{ij}) \) for all \( i,j \in I \), there exists \( m \in M^*(\Omega) \) such that

\[
m / m_i \in A^*(U_i), \text{ for all } i \in I.
\]
\end{theorem}

Equivalently: The map \( \text{div}: M^*(\Omega) \rightarrow D(\Omega) \) is surjective. That is, subject to the requirement that pole and zero sets are discrete, we may construct a meromorphic function on \( \Omega \) with prescribed zeros and poles of given orders.

\begin{proof}
By taking a refinement of the cover U, it is no loss of generality to assume that each \( U_i \in U \) is convex (for example, an open disc).

Given \( i,j \in I \), set \( h_{ij} = m_i / m_j \). Fix a branch of log \( h_{ij} \) on \( U_{ij} \) and define

\[
c_{ijk} = \frac{1}{2\pi i} (\log h_{ij} + \log h_{jk} + \log h_{ki}).
\]

Since \( h_{ij} h_{jk} h_{ki} \equiv 1 \), \( c_{ijk} \in \mathbb{Z} \) for all \( i,j,k \) (note that \( U_{ijk} \) is convex and therefore connected and so \( c_{ijk} \) is constant on \( U_{ijk} \)).

Suppose that we can choose the branches of log \( h_{ij} \) so that \( c_{ijk} = 0 \) for all \( i,j,k \in I \). Then if we set \( f_{ij} = \log h_{ij} \in A(U_{ij}) \), we see that the conditions of Theorem 1.3.2' are satisfied. Hence there exist \( g_i \in A(U_i) \) such that for all \( i,j \in I \)

\[
\log h_{ij} = g_j - g_i \text{ on } U_{ij}.
\]

Now define \( a_i = \exp(g_i) \in A^*(U_i) \). Clearly \( a_j / a_i = h_{ij} = m_i / m_j \) and so

\[
m_i a_i = m_j a_j \text{ on } U_{ij}.
\]

Hence we can define \( m \in M^*(\Omega) \) by taking \( m|U_i = m_i a_i \). Clearly \( m \) satisfies the required conditions.

To complete the proof it is sufficient to show that we can choose the branches of log \( h_{ij} \) so that \( c_{ijk} = 0 \) for all \( i,j,k \). This is essentially a topological problem. First we observe that \( \{c_{ijk}\} \) defines a class in \( H^2(U,\mathbb{Z}) \) (For those unfamiliar with Čech theory we refer to Chapter 6). Now any finite intersection \( U \) of open sets of the cover \( U \) is convex and so \( H^p(U,\mathbb{Z}) = 0 \), \( p \neq 0 \). Hence by Leray's theorem (Theorem 6.3.16)

\[
H^2(U,\mathbb{Z}) \cong H^2(\Omega,\mathbb{Z}).
\]

But \( H^2(\Omega, \mathbb{Z}) = 0 \) for all open domains in \( \mathbb{C} \) (in fact for all non-compact oriented 2-manifolds - see the remarks following this proof). Hence \( \{c_{ijk}\} \) is a coboundary and there exist integers \( n_{ij}, i,j \in I \), such that

\[
c_{ijk} = n_{ij} + n_{jk} + n_{ki}, \text{ for all } i,j,k \in I.
\]

Now define a new branch of log \( h_{ij} \) by subtracting \( 2\pi i n_{ij} \) from the original choice. For the new choice of branches we have \( c_{ijk} = 0 \), for all \( i,j,k \).
\end{proof}

Notice that the proof of Theorem 1.3.3 actually proves the slightly stronger Theorem 1.3.3'. Let \( h_{ij} \in A^*(U_{ij}) \) satisfy the cocycle conditions
\[
h_{ij} h_{jk} h_{ki} = 1 \text{ on } U_{ijk} \text{ and } h_{ij} = h^{-1}_{ji} \text{ on } U_{ij}, i,j,k \in I.
\]
Then there exist \( h_i \in A^*(U_i) \) such that for all \( i,j \)
\[
h_{ij} = h_j h^{-1}_{i} \text{ on } U_{ij}.
\]

Remarks

1. The reader should observe the similarity between the proofs of Theorems 1.3.2 and 1.3.3. Indeed the last part of the above proof can be written as a problem in \( C^\infty \) functions if we note that \( c_{ijk} = 0 \) for all \( i,j,k \) iff there exists \( b_i \in C^\infty(U_i) \) such that log \( h_{ij} = b_j / b_i \). Later, in Chapter 6, we develop machinery that handles all arguments of this type in a particularly simple and elegant way.

2. The explicit use of cohomology can of course be avoided in the proof of Weierstrass' theorem. See, for example, the proof given in Hörmander [1]. The reader might wish to attempt a direct construction of the integers \( n_{ij} \) without using any general facts about the cohomology of \( 2 \)-manifolds.

3. For the vanishing of \( H^2(\Omega, \mathbb{Z}) \), we note that \( H^2(\Omega, \mathbb{Z}) \cong H_2(\Omega, \mathbb{Z}) \) (computation using universal coefficient theorems and the fact that \( \Omega \) is a \( 2 \)-manifold). Since \( \Omega \) is assumed oriented, Poincaré duality implies that \( H_2(\Omega, \mathbb{Z}) \cong H^0_C(\Omega, \mathbb{Z}) \), where \( C \) denotes compact supports. Since \( \Omega \) is non-compact, \( H^0_C(\Omega, \mathbb{Z}) = 0 \). For further details and references we refer the reader to the appendix in Milnor and Stasheff [1].

\begin{corollary}
Let \( m \in M(\Omega) \). Then there exist \( f,g \in A(\Omega) \) such that

\begin{enumerate}
\item Off the pole set of \( m \), \( m = f / g \).
\item \( Z(m) = Z(f) \), \( P(m) = Z(g) \) and \( Z(f) \cap Z(g) = \emptyset \).
\end{enumerate}
\end{corollary}

\begin{proof}
Suppose \( m \) has poles \( z_j \) with orders \( p_j \). Then, by Weierstrass' theorem there exists \( g \in A(\Omega) \) such that \( \text{div}(g) = \sum_{j} p_j \cdot z_j \). Since \( \text{div}(m g) \geq 0 \), \( m g \in A(\Omega) \). Taking \( f = m g \), it is clear that \( f \) and \( g \) satisfy the conditions of the corollary.
\end{proof}

\begin{remark}
\( f \) and \( g \) are unique up to multiplication by elements of \( A^*(\Omega) \).
\end{remark}

\begin{corollary}
There exists \( f \in A(\Omega) \) which cannot be extended as an analytic or meromorphic function to any open subset strictly containing \( \Omega \).
\end{corollary}

\begin{proof}
Let \( d(z,\partial \Omega) \) denote the distance between \( z \in \mathbb{C} \) and the boundary of \( \Omega \), \( \partial \Omega \). Define

\[
X = \left\{ \left( \frac{p}{2^n}, \frac{q}{2^n} \right) \in \Omega: p,q,n \in \mathbb{Z}, n \geq 0 \text{ and } d\left( \left( \frac{p}{2^n}, \frac{q}{2^n} \right), \partial \Omega \right) < 2^{-n+2} \right\}
\]

\( X \) is a discrete subset of \( \Omega \) and every point of \( \partial \Omega \) is the limit of some subsequence of \( X \). By Weierstrass' theorem there exists \( f \in A(\Omega) \) such that

\[
\text{div}(f) = \sum_{x \in X} 1 \cdot x.
\]

Since the zeros of an analytic or meromorphic function are isolated, we see that \( f \) cannot be extended to any domain strictly containing \( \Omega \). Indeed, such a domain would contain at least one point of \( \partial \Omega \) and this point would then be a non-isolated zero of the extension.
\end{proof}

\begin{remark}
The proof of Corollary 1.3.5 actually shows that we can find an analytic function on \( \Omega \) which cannot be continued (locally) across any point of the boundary of \( \Omega \).
\end{remark}

\begin{xcb}{Exercise}
Let \( \{z_i : i \geq 0\} \) be a discrete subset of \( \Omega \subset \mathbb{C} \) and suppose that for each \( i \) we are given a polynomial \( P_{i}(z) = \sum_{j=0}^{k_i} a_{ij}(z - z_j)^{j} \). Show that there exists an analytic function \( f \) on \( \Omega \) such that the Taylor series of \( f \) at each \( z_i \) agrees with \( P_{i} \) to order \( k_i \). (Hint: Find a meromorphic function \( \mathfrak{m} \) on \( \Omega \) such that at each \( z_i \) the principal part of \( \mathfrak{m} \) equals \( (z - z_i)^{-k_i-1} P_{i}(z) \). Then use Weierstrass' theorem).
\end{xcb}

\section{Riemann surfaces}

In sections 4 and 5 we make a preliminary investigation of possible generalisations of the results of §3 to Riemann surfaces. As we shall see both the topology and complex structure of a Riemann surface play a crucial role as is also the case in the theory of higher dimensional complex manifolds.

Let \( U \) and \( V \) be open subsets of \( \mathbb{C} \) and \( f: U \to V \). We shall say that \( f \) is biholomorphic if \( f \) is a homeomorphism onto \( V \) and both \( f \) and \( f^{-1} \) are analytic. We remark that, by the inverse function theorem, for \( f \) to be biholomorphic it is sufficient for \( f \) to be bijective and everywhere analytic with non-vanishing derivative.

Before recalling the definition of a Riemann surface, we wish to stress that all topological spaces considered in this book will be Hausdorff and paracompact. Unless the contrary is clearly indicated they will also be connected. The Hausdorff and paracompactness assumptions imply metrizability and, together with connectedness, imply separability (see, for example, Matsushima [1]).

A chart on a topological space \( M \) consists of a pair \( (U,\phi) \) where \( U \) is an open subset of \( M \) and \( \phi \) is a homeomorphism of \( U \) onto an open subset of Euclidean space.

\begin{definition}
Let \( M \) be a topological space. Suppose that we are given a set \( A = \{(U_{i},\phi_{i}) : i \in I\} \) of charts on \( M \) satisfying

\begin{enumerate}
\item \( \{U_i\} \) is an open cover of M.
\item For each \( i \in I \), \( \phi_i \) is a homeomorphism of \( U_i \) onto an open subset of \( \mathbb{C} \).
\item For all \( i,j \in I \), \( \phi_i \phi_j^{-1} \) is a biholomorphism of \( \phi_j (U_{ij}) \) with \( \phi_i (U_{ij}) \).
\end{enumerate}

Then A is called a (complex analytic) atlas on M and M, together with the atlas A, is called a Riemann surface.
\end{definition}

Remarks.

1. We refer to charts \( (U,\phi) \in A \) as complex analytic charts on M or just charts on M if the analyticity is clear from the context.

2. If A is an atlas on M, we shall generally assume that A is maximal in the sense that if \( (U,\phi) \) is a chart on M such that \( \phi \phi_i^{-1} \) is biholomorphic for all \( (U_i, \phi_i) \in A \) then \( (U,\phi) \in A \). Of course, every atlas can be completed to a maximal atlas.

Before giving examples and motivation for the study of Riemann surfaces we shall generalise some of the definitions of §§1,2.

Suppose that M is a Riemann surface with atlas A. Let W be an open subset of M. If $f: W \rightarrow \mathbb{C}$, we say f is analytic if \( f \phi^{-1} \in A (\phi(U \cap W)) \) for all \( (U,\phi) \in A \). We let A(W) denote the ring of analytic functions on W. We say that m is a meromorphic function on W if \( m \phi^{-1} \in M (\phi(U \cap W)) \) for all \( (U,\phi) \in A \) (a more careful definition can be made along the lines of Definition 1.2.1). We let M(W) denote the ring of meromorphic functions on W and \( M^*(W) \), \( A^*(W) \) denote the groups of units in M(W), A(W) respectively. Note that if W is connected M(W) is a field.

Let $m \in M^*(M)$. We define the zero set Z(m) of m by

\[
Z(m) = \bigcup \phi^{-1} Z(m \phi^{-1}),
\]

where the union is over all charts \( (U,\phi) \in A \). We similarly define the pole set P(m) of m. We define the order of m at z, ord(m,z) to be $ord(m \phi^{-1}, \phi(z))$, where \( (U,\phi) \in A \) contains z, and the divisor of m, div(m), by

\[
\text{div}(m) = \sum_{z \in M} \text{ord}(m,z) \cdot z
\]

We leave it to the reader to verify that Z(m) and P(m) are well-defined, discrete, disjoint subsets of M; that $M \setminus P(m)$ is the largest subset of M on which m is analytic; that ord(m,z) does not depend on the choice of chart containing z and that if M is compact, Z(m) and P(m) are finite.

Suppose $N$ is another Riemann surface with atlas $B$ and $f: M \rightarrow N$. We say that $f$ is analytic or holomorphic if \( \psi f \phi^{-1}: \phi(U \cap f^{-1}(V)) \rightarrow \mathbb{C} \) is analytic for all \( (U,\phi) \in A \), \( (V,\psi) \in B \) and that f is biholomorphic if f is bijective and both f and \( f^{-1} \) are analytic. If there exists a bihomorphic map between M and N, we say that M and N are biholomorphically equivalent or analytically isomorphic.

Examples.

1. The \textit{Riemann sphere}. The Riemann sphere is defined to be the 1-point compactification of \( \mathbb{C} \), \( \mathbb{C} \cup \{ \infty \} \), and we shall denote it by either \( S^2 \) or \( P^1(\mathbb{C}) \) (see Chapter 4 for the second notation). We define an atlas \( \{(U,\phi),(V,\psi)\} \) on \( S^2 \) by taking \( U = \mathbb{C} \), \( V = \mathbb{C}^* \cup \{ \infty \} \) (here \( \mathbb{C}^* \) denotes non-zero complex numbers) and \( \phi(z) = z \); \( \psi(z) = 1/z \), \( \psi(\infty) = 0 \).

Suppose \( \Omega \) is an open subset of \( \mathbb{C} \) and \( m \in M(\Omega) \). Then m induces an analytic map \( \tilde{m}: \Omega \rightarrow S^2 \) if we take \( \tilde{m}|\Omega \setminus P(m) = m \) and set \( \tilde{m}(p) = \infty \), for all \( p \in P(m) \). The verification that \( \tilde{m} \) is analytic is easy using our explicit atlas for \( S^2 \).

2. If $M$ is a \textit{simply connected} Riemann surface then $M$ is biholomorphic to precisely one of: The Riemann sphere, the open unit disc, the complex plane. This is the fundamental (and difficult!) theorem of Riemann surfaces known as the *Uniformization Theorem* (for a proof and more details we refer to Ahlfors and Sario [1], Siegel [1], Springer [1]).

3. Let $f: M \rightarrow N$ be a local homeomorphism between topological spaces M and N. That is, for each $x \in M$ there exists an open neighbourhood U of x such that \( f(U) \) is an open subset of N and \( f|U \) maps U homeomorphically onto \( f(U) \). Suppose that N is a Riemann surface with atlas A. Then f induces the structure of a Riemann surface on M in such a way that f becomes a holomorphic map which is locally biholomorphic. Indeed, suppose U is an open subset of M which is mapped homeomorphically by f onto an open subset of \( N \). Suppose \( (W, \phi) \in A \) and \( W \supset f(U) \). Then \( (U, \phi f) \) is a chart on \( M \). The set of all charts on \( M \) constructed in this way is easily seen to define a complex analytic atlas on \( M \) with respect to which \( f \) is locally biholomorphic.

4. By examples 2 and 3, the simply connected covering surface of a Riemann surface has the natural structure of a Riemann surface and is biholomorphic to one of: the Riemann sphere, the open unit disc, the complex plane (see also Chapter 4).

5. If \( M \) is a compact Riemann surface, \( M \) is diffeomorphic to a compact orientable surface. Such surfaces are classified, up to diffeomorphism by their genus or number of handles (see Hirsch [1]). However, genus does not classify \( M \) up to biholomorphic equivalence (unless \( M \) is simply connected). For a specific example, we refer to Chapter 4, §4.

6. Suppose \( \pi : M \rightarrow \mathbb{C} \) is a local homeomorphism. Then \( M \) has the structure of a non-compact Riemann surface with respect to which \( \pi \) is locally biholomorphic. The pair \( (M, \pi) \) is called a Riemann domain or spreading of \( M \) over \( \mathbb{C} \). Riemann domains occur as the domains of maximal analytic continuation of analytic functions defined on open subsets of \( \mathbb{C} \). Indeed, if \( \Omega \) is a domain in \( \mathbb{C} \) and \( f \in A(\Omega) \), it is not hard to construct a Riemann domain \( (M_f, \pi) \) such that \( f \) extends analytically to \( M_f \) and \( (M_f, \pi) \) is the maximal Riemann domain to which \( f \) can be continued analytically (The converse is also true but much harder: Any Riemann domain is the domain of maximal analytic continuation of an analytic function. This result, due to Behnke and Stein [1], generalises Corollary 1.3.5 of the Weierstrass theorem. See also R. Narasimhan [1] and Gunning and Rossi [1]). We give one explicit example here and leave the general construction to Chapter 6.

Let Log denote the principal branch of the logarithm defined off the negative real axis. We let \( M \subset \mathbb{C}^2 \) denote the set of all points \( (z, \log |z| + \arg z) \), \( z \in \mathbb{C}^* \), where \( \arg z \) is a value of the argument of \( z \). If we let \( \pi \) denote the restriction to \( M \) of the projection of \( \mathbb{C}^2 \) on the first factor, we may easily verify that \( (M, \pi) \) is a Riemann domain. Furthermore, \( M \) is the maximal Riemann domain to which Log extends analytically. Moreover, if we let \( \log : M \rightarrow \mathbb{C} \) denote the restriction to \( M \) of the projection of \( \mathbb{C}^2 \) on the second factor, we see that log is the analytic extension of Log to M and that if \( y \in \pi^{-1}(z) \), \( z \in \mathbb{C}^* \), then \( \log(y) = \log|z| + i\phi \), where \( \phi \) is a value of arg \( z \).

7. There is no natural way of compactifying the Riemann domain associated to the logarithm that we constructed in the previous example. This is a reflection of the transcendental (non-algebraic) character of log (see Chapter 7). In the case of algebraic functions we can always compactify the Riemann domain associated to the function in a particularly nice way. Here we shall consider a simple example and refer the reader to Ahlfors [1] or Siegel [1] for the general theory.

Let us regard \( P(y,z) = y^3 - z^2 \) as a polynomial in \( y \) with coefficients depending on \( z \). The zero set of \( P \) defines a subset \( X \) of \( \mathbb{C}^2 \). Now \( X \) will not be a complex submanifold of \( \mathbb{C}^2 \) as there is a singularity or branch point at \( z = 0 \) (see Chapter 4 for terminology). Let \( \pi: X \to \mathbb{C} \) denote the restriction to \( X \) of the projection of \( \mathbb{C}^2 \) on the \( z \)-axis. If \( X_p \) denotes the 1-point compactification of \( X \) then, if we define \( \pi(\omega) = \infty \), \( \pi \) extends continuously to \( X_p \) and \( \pi: X_p \to S^2 \) is a 3-fold branched cover of \( S^2 \) with branch points at 0 and \( \infty \). In particular, if we set \( X^* = X \setminus \{0,\infty\} \), \( \pi: X^* \to \mathbb{C}^* \) is a local homeomorphism and \( \pi^{-1}(z) \) contains precisely three points for all \( z \in \mathbb{C}^* \). Now, by example 3, \( (X^*, \pi) \) is a Riemann domain and we denote its atlas by \( A^* \). We now construct a complex analytic atlas for the whole of \( X_p \) with respect to which \( \pi \) is analytic. Let \( D \) denote the open unit disc centre zero in the \( \zeta \)-plane. We define

\[
\phi: D \to X_p \text{ by } \phi(\zeta) = (\zeta^2, \zeta^3)
\]

\[
\psi: D \to X_p \text{ by } \psi(\zeta) = (\zeta^{-2}, \zeta^{-3}), \zeta \neq 0
\]

\[
= \infty, \zeta = 0.
\]

Notice that \( \phi \) and \( \psi \) are homeomorphisms onto open neighbourhoods of \( \pi^{-1}(0) \) and \( \infty \) respectively. As atlas on \( X_p \) we take

\[
\{(\phi(D), \phi^{-1}), (\psi(D), \psi^{-1})\} \cup A^*.
\]

We leave it to the reader to verify that this atlas defines on \( X_p \) the structure of a compact Riemann surface, biholomorphic to the Riemann sphere, with respect to which \( \pi \) is analytic.

Let p: \( X_p \to S^2 \) denote the restriction to \( X_p \) of the projection of \( \mathbb{C}^2 \) onto the second factor where we have defined \( p(\infty) = \infty \). Then p is a meromorphic function on \( X_p \) with a pole of order 3 at \( \infty \) and a zero of order 3 at 0. p gives all the roots of \( y^3 = z^2 \) in the sense that 
\[ \{p(x): x \in \pi^{-1}(z)\} \] is the set of all roots of \( y^3 = z^2 \) for all \( z \in \mathbb{C} \).

For the remainder of this section we wish to investigate the possible generalisation of Weierstrass' theorem to compact Riemann surfaces. So from now on assume M is a compact Riemann surface.

\begin{lemma}
Every analytic function on M is constant.
\end{lemma}

\begin{proof}
If \( f \in A(M) \), \( |f| \) has a maximum on M, say at \( z_0 \). A simple application of the maximum modulus theorem shows that f is constant in some neighbourhood of \( z_0 \). Uniqueness of analytic continuation then implies that f is constant on M.
\end{proof}

In view of this Lemma it becomes a matter of some importance to prove that there exist plenty of non-constant meromorphic functions on M. Specifically: Let \( d \in D(M) \). Under what conditions on d, if any, do there exist \( m \in M^*(M) \) such that \( \text{div}(m) = d \)?

For the Riemann sphere this question is easily solved:
If \( d = \sum_{i=1}^n p_i \cdot z_i \in D(S^2) \), then there exists \( m \in M^*(S^2) \) with \( \text{div}(m) = d \) if and only if \( \sum_{i=1}^n p_i = 0 \).

Indeed, if \( z_1, \ldots, z_{n-1} \in \mathbb{C}, z_n = \infty \), we may define
\[
m(z) = \prod_{i=1}^{n-1} (z - z_i)^{p_i}, \quad z \in \mathbb{C}.
\]
m extends to a meromorphic function on \( S^2 \) with singularities of order \( p_i \) at \( z_i \), $1 \leq i \leq n-1$, and of order \( -\sum_{i=1}^{n-1} p_i \) at $\infty$.

The converse is included in Lemma 1.4.3. below.

In future if \( d = \sum_{z \in M} p(z) \cdot z \) is a divisor on M, we set
\[
\deg(d) = \sum_{z \in M} p(z) \in \mathbb{Z}
\]
and call \( \deg(d) \) the degree of d.

We have the following topological restriction on divisors of meromorphic functions.

\begin{lemma}
A necessary condition for \( d \in D(M) \) to be the divisor of a meromorphic function is that \( \deg(d) = 0 \).
\end{lemma}

\begin{proof}
Let \( m \in M^*(M) \) and set \( \text{div}(m) = \sum_{i=1}^n n_i z_i \). For \( i = 1, \ldots , n \) choose charts \( (U_i, \phi_i) \) containing \( z_i \) and such that \( \phi_i(U_i) \) contains \( \overline{D}_1(0) \). Set \( D_i = \phi_i^{-1}(\overline{D}_1(0)) \) and \( \gamma_i = \partial D_i = \phi_i^{-1}(\partial \overline{D}_1(0)) \). We shall assume that the charts are chosen so that the sets \( D_i \) are mutually disjoint.

Log \( m \) is defined up to integer multiples of \( 2\pi i \) on \( M' = M \setminus \cup D_i \).

Hence the 1-form \( \phi = d(\log m) \) is well defined on \( M' \). By Stokes' theorem

\[
\int_{M'} d\phi = \sum_{i=1}^n \int_{\gamma_i} \phi.
\]

But \( d\phi = d^2(\log m) = 0 \), and so the left hand integral must vanish. Now

\[
\sum_{i=1}^n \int_{\gamma_i} \phi = \sum_{i=1}^n \int_{|z|=1} \partial (\log m_i)/\partial z \, dz, \quad \text{where } m_i = m \phi_i^{-1}
\]

\[
= \sum_{i=1}^n \int_{|z|=1} m_i'/ m_i \, dz
\]

\[
= \sum_{i=1}^n n_i, \quad \text{by the residue theorem}.
\]

Hence \( \deg(d) = 0 \).
\end{proof}

The reader is warned that the vanishing of the degree of d is not generally sufficient for d to be the divisor of a meromorphic function (see the end of §5 and also Chapter 4, §4).

In view of the importance of Theorem 1.3.1 in the proofs of the Mittag-Leffler and Weierstrass theorems it is natural to seek a generalisation to Riemann surfaces. The first difficulty we encounter is that of finding a suitable definition of \( \partial / \partial \bar{z} \) for a Riemann surface. Indeed, suppose \( f \in C^\infty(M) \) and \( (U,\phi) \) and \( (V,\psi) \) are complex analytic charts on \( M \). In general, \( \partial (f \phi^{-1}) / \partial \bar{z} \neq \partial (f \psi^{-1}) / \partial \bar{z} \) and so we cannot expect to define \( \partial / \partial \bar{z} \) as an endomorphism of \( C^\infty(M) \). The correct generalisation of \( \partial / \partial \bar{z} \) involves the introduction of "twisted" functions and tensor fields on M. These concepts are most easily discussed in the general framework of vector bundles and we shall defer further consideration of the construction of meromorphic functions until we have developed sufficient of the elementary theory of vector bundles in §5. The reader who is largely unfamiliar with the theory of vector bundles may prefer to go straight to Chapter 2 and return to section 5 after the end of Chapter 4.

\begin{xcb}{Exercises}
\begin{enumerate}
\item Construct the (compact) Riemann surface of \( z^2 = (y - a)(y - b) \), $a \neq b$, and show that it is biholomorphic to the Riemann sphere. (Hint: prove that the Riemann surface is simply connected).

\item Construct the Riemann surface of \( y^2 = (z - a)(z - b)(z - c) \), a,b,c distinct, and show that it is diffeomorphic to the two dimensional real torus.

\item Formalise and prove a version of the Mittag-Leffler theorem valid for the Riemann sphere.

\item Let f: M → N be a holomorphic map between Riemann surfaces. Show
   \begin{enumerate}
   \item Either f is constant or f(M) is an open subset of N.
   \item In case M is compact, either f is constant or f(M) = N.
   \item Given \( z_0 \in N \), \( f^{-1}(z_0) \) is a discrete subset of M (possibly empty).
   \end{enumerate}

\item Let $f: M \to N$ be a non-constant map between Riemann surfaces $M$ and $N$ and suppose $d \in D(N)$. Show how to define $f^*(d) \in D(M)$ and verify that if $m \in M^*(N)$ then $\text{div}(mf) = f^*\text{(div}(m))$.
\end{enumerate}
\end{xcb}

\section{Vector bundles}

The first part of this section constitutes a summary of the theory of real and complex vector bundles over a topological space. For more comprehensive treatments the reader may refer to Abraham and Marsden [1], Husmoller [1] or Lang [1].

\begin{definition}
Let X be a topological space. An n-dimensional real vector bundle E over X consists of a topological space E and continuous map \( \pi: E \to X \) which satisfy

\begin{enumerate}
\item The fibre \( \pi^{-1}(x) = E_x \) has the structure of a real n-dimensional vector space for every \( x \in X \).
\item There exists an open cover \( \{U_i : i \in I\} \) of X and homeomorphisms \( \theta_i : \pi^{-1}(U_i) \to U_i \times \mathbb{R}^n \) such that for all \( i \in I, x \in U_i \), \( \theta_i \) maps \( E_x \) linearly and isomorphically onto \( \{x\} \times \mathbb{R}^n \).
\end{enumerate}

We denote the vector bundle by \( \pi: E \to X \) or just E.

The topological space E is called the total space of the vector bundle, X is called the base space and \( \pi \) the projection map. The maps \( \theta_i \) are called trivialisations of E.
\end{definition}

Example 1. Let \( \pi: X \times \mathbb{R}^n \to X \) denote projection on the first factor. Then \( \pi: X \times \mathbb{R}^n \to X \) is an n-dimensional vector bundle over X called the trivial n-dimensional vector bundle over X. We often denote the trivial bundle over X with fibre \( \mathbb{R}^n \) by \( \mathbb{R}^n \).

Condition (b) on a vector bundle implies that a vector bundle is locally a trivial bundle. Globally it may be twisted.

\begin{definition}
A (continuous) section of the vector bundle \( \pi: E \to X \) is a (continuous) map \( \mathbf{s}: X \to E \) such that \( \pi \mathbf{s} = \text{identity} \). That is, \( \mathbf{s}(x) \in E_x \) for all \( x \in X \).
\end{definition}

Example. If f: \( X \to \mathbb{R}^n \), then the graph map \( x \mapsto (x,f(x)) \) is a section of the trivial bundle \( X \times \mathbb{R}^n \). Conversely, every section of the trivial bundle \( X \times \mathbb{R}^n \) defines an \( \mathbb{R}^n \)-valued function on X.

We shall let \( C^0(E) \) denote the set of all continuous sections of E and remark that \( C^0(E) \) has the structure of a real vector space with addition and scalar multiplication induced from the vector space structure on the fibres of E.

If \( \mathbf{s} \) is a section of the vector bundle E, then \( \theta_i \mathbf{s}: U_i \to U_i \times \mathbb{R}^n \) is the graph map of a function \( S_i: U_i \to \mathbb{R}^n \). Thus we may think of sections of E as locally being \( \mathbb{R}^n \)-valued functions. Globally, they may be "twisted" \( \mathbb{R}^n \)-valued functions.

Example. Let \( S^1 \times \mathbb{R} \) denote the twisted cylinder. \( S^1 \times \mathbb{R} \) is homeomorphic to the Mobius band and may be explicitly parametrized as the subset \[ \{ (\cos \theta, \sin \theta, t \cos \theta/2, t \sin \theta/2) : \theta \in [0, 2\pi), t \in \mathbb{R} \} \] of \( \mathbb{R}^4 \). The space \( S^1 \times \mathbb{R} \) has the structure of a 1-dimensional real vector bundle over \( S^1 \) if we define projection onto \( S^1 \) in the obvious way.

\begin{figure}
% \includegraphics{filename}
\caption{Image of s}
\label{fig:1}
\end{figure}

We observe that every continuous section of \( S^1 \times \mathbb{R} \) is zero at at least one point in sharp contrast to what happens for sections of \( S^1 \times \mathbb{R} \). Thus although locally, twisted and untwisted functions on \( S^1 \) possess the same properties, their global behaviour may be rather different.

We shall now give an alternative description of vector bundles in terms of transition functions. We follow the notation of Definition 1.5.1. For each \( i \in I, x \in U_i \), we let

\[
\theta_{ix} : E_x \rightarrow \mathbb{R}^n
\]

denote the restriction of \( \theta_i \) to \( E_x \) composed with the projection on \( \mathbb{R}^n \).

Define

\[
\theta_{ij} : U_{ij} \rightarrow \mathrm{GL}(\mathbb{R}^n)
\]

by \( \theta_{ij}(x) = \theta_{ix} \theta_{jx}^{-1} \), \( x \in U_{ij} \), \( i,j \in I \) (\( \mathrm{GL}(\mathbb{R}^n) \) denotes the group of linear isomorphisms of \( \mathbb{R}^n \) which is commonly given the alternative notation \( \mathrm{GL}(n,\mathbb{R}) \)).

It may easily be verified that the \( \theta_{ij} \) are continuous maps which satisfy the cocycle conditions

\[
\begin{cases}
\theta_{ij} = \theta_{ji}^{-1} \\
\theta_{ij} \cdot \theta_{jk} = \theta_{ik}
\end{cases}
\quad \text{(A)}
\]

(Inversion and multiplication are in \( \mathrm{GL}(\mathbb{R}^n) \)). The \( \theta_{ij} \) are called transition functions of the vector bundle E. Suppose s is a section of E and \( S_i : U_i \rightarrow \mathbb{R}^n \) is the local representation of s on \( U_i \) as described above. Then for all \( i,j \)

\[
\theta_{ij} S_j = S_i \text{ on } U_{ij} \quad \text{(B)}
\]

Conversely, any set of functions \( S_i : U_i \rightarrow \mathbb{R}^n \) satisfying (B) defines a unique section of E.

If E and F are vector bundles over X we shall say that a continuous map A: \( E \rightarrow F \) is a vector bundle map if for all \( x \in X \), \( A(E_x) \subseteq F_x \) and \( A|E_x \) is linear. If A is bijective and \( A^{-1} \) is a vector bundle map we shall say that A is a vector bundle isomorphism and that E and F are isomorphic vector bundles.

Suppose that E, F have trivialising maps \( \theta_i, \eta_i \) respectively, relative to some (common) cover \( \{U_i : i \in I\} \) of X. Then for each \( i \in I \), A induces a continuous map \( A_i : U_i \times \mathbb{R}^n \rightarrow U_i \times \mathbb{R}^n \) given by \( A_i = \eta_i A \theta_i^{-1} \). In turn, \( A_i \) induces a continuous map

\[
a_i : U_i \rightarrow L(\mathbb{R}^n, \mathbb{R}^n)
\]

such that for all \( i,j \)

\[
\eta_{ij} a_j = a_i \theta_{ij} \text{ on } U_{ij} \quad \text{(C)}
\]

Conversely, a family of continuous maps \( a_i: U_i \rightarrow L(\mathbb{R}^n, \mathbb{R}^m) \) determines a vector bundle map from E to F provided that for all \( i,j \)

\[
\eta_{ij} a_j = a_i \theta_{ij} \text{ on } U_{ij}.
\]

Notice that E will be a trivial vector bundle (that is, isomorphic to a trivial vector bundle) if and only if there exist \( a_i: U_i \rightarrow \mathrm{GL}(\mathbb{R}^n) \) such that for all \( i,j \)

\[
a_j = a_i \theta_{ij} \text{ on } U_{ij}.
\]

Notation. Let \( \mathbb{R}^{m'} \) denote the real dual of \( \mathbb{R}^m \). If

A \( \in L(\mathbb{R}^m, \mathbb{R}^n) \), then the transpose of A, A', is the linear map from \( \mathbb{R}^{n'} \rightarrow \mathbb{R}^{m'} \) characterised by \( A'(\phi)(e) = \phi(A(e)) \), \( \phi \in \mathbb{R}^{n'} \), \( e \in \mathbb{R}^m \).

Example 4. Let E be an n-dimensional real vector bundle with transition functions \( \theta_{ij} \). We define \( \theta_{ij}' : U_{ij} \rightarrow \mathrm{GL}(\mathbb{R}^{n'}) \) by

\[
\theta_{ij}'(x) = [\theta_{ij}(x)]^{-1}, \quad x \in U_{ij}.
\]

The \( \theta_{ij}' \) satisfy the cocycle condition (A) and so are the transition functions for a vector bundle which we call the dual vector bundle of E and denote by \( E' \). We remark that the trivialisations \( \theta_i' \) of \( E' \) map naturally to \( U_i \times \mathbb{R}^{n'} \) (rather than \( U_i \times \mathbb{R}^n \)).

Motivated by the previous example, we shall now extend the possible "fibre models" of real vector bundles to include any finite combination of direct sum, tensor product, exterior and symmetric power of \( \mathbb{R}^n \) and \( \mathbb{R}^{n'} \). All these vector space operations extend immediately to vector bundles over a fixed base space \( X \). We give one example to illustrate the method and refer the reader to the references for more details. Suppose that \( E \) and \( F \) are vector bundles over \( X \) with transition functions \( \theta_{ij}: U_{ij} \rightarrow \mathrm{GL}(\mathbb{R}^n) \) and \( \eta_{ij}: U_{ij} \rightarrow \mathrm{GL}(\mathbb{R}^n) \) respectively. \( \wedge^r F \otimes (E \otimes (S^s F)) \) will then denote the vector bundle over \( X \) with fibre model \( \wedge^r \mathbb{R}^n \otimes (\mathbb{R}^m \otimes (S^s \mathbb{R}^n)) \) and transition functions \( \psi_{ij}: U_{ij} \rightarrow \mathrm{GL}(\wedge^r \mathbb{R}^n \otimes (\mathbb{R}^n \otimes (S^s \mathbb{R}^n))) \) defined by 
\[
\psi_{ij}(x) = (\wedge^r \eta_{ij}(x)) \otimes (\theta_{ij}(x) \otimes (S^s \eta_{ij}(x))), \quad x \in U_{ij}.
\]
(\( \wedge^r \) and \( S^s \) denote the operations of rth exterior and sth symmetric power respectively.)

Suppose that \( X \) is a differential manifold (with \( \mathcal{C}^\infty \) structure). We can define smooth (that is \( \mathcal{C}^\infty \)) real vector bundles over \( X \) in the obvious way and everything we have said above generalises immediately to the smooth case. If \( E \) is a smooth vector bundle over \( X \), we let \( \mathcal{C}^r(E) \) denote the vector space of \( \mathcal{C}^r \) sections of \( E \), \( 1 \leq r \leq \infty \).

Example 5. Let \( M \) be an m-dimensional differential manifold with smooth atlas \( \{(U_i,\phi_i): i \in I\} \). We define \( \mathcal{C}^\infty \) maps \( \phi_{ij}: U_{ij} \rightarrow \mathrm{GL}(\mathbb{R}^m) \) by \( \phi_{ij}(x) = D(\phi_i \phi_j^{-1})(\phi_j(x)) \). The \( \phi_{ij} \) are the transition functions for the (real) tangent bundle of \( M \) which we denote in the sequel by \( \mathcal{T}_M \). The (real) cotangent bundle of \( M \), \( \mathcal{T}_M^* \), is the dual bundle of \( \mathcal{T}_M \) with transition functions \( \phi_{ij}' \) (See Abraham and Marsden [1], Chillingworth [1] and Lang [1] for more details).

We now turn our attention to vector bundles with fibre a complex vector space. An m-dimensional complex vector bundle \( E \) over \( X \) is defined exactly as in Definition 1.5.1 except that "real" and "\( \mathbb{R}^n \)" are everywhere replaced by "complex" and "\( \mathbb{C}^m \)" respectively. We leave the writing out of the formal definitions of complex vector bundles, sections, transition functions, etc. to the reader. Sections of an m-dimensional complex vector bundle will locally be \( \mathbb{C}^m \)-valued functions: globally they may be twisted.

One important feature of complex vector bundles is the greater complexity of the fibre models which involve not only duals but also conjugates. But first we need to review some complex linear algebra.

If \( J: \mathbb{C}^m \rightarrow \mathbb{C}^m \) denotes scalar multiplication by \( i \), then \( J \) is a \( \mathbb{C} \)-linear isomorphism of \( \mathbb{C}^m \) satisfying \( J^2 = -I \). The usual complex structure on \( \mathbb{C}^m \) may be defined in terms of \( J \) and the underlying real structure on \( \mathbb{C}^m \) by

\[
(a + ib)Z = aZ + bJ(Z), \quad a,b \in \mathbb{R}, \quad Z \in \mathbb{C}^m.
\]

The conjugate complex structure on \( \mathbb{C}^m \) is defined by

\[
(a + ib)Z = aZ - bJ(Z), \quad a,b \in \mathbb{R}, \quad Z \in \mathbb{C}^m.
\]

We denote the resulting complex vector space by \( \overline{\mathbb{C}}^m \). Suppose \( A \in L(\mathbb{C}^m, \mathbb{C}^n) \), we define \( \overline{A} \in L(\overline{\mathbb{C}}^m, \overline{\mathbb{C}}^n) \) by simply requiring that \( A = \overline{A} \) on the common underlying set of \( \mathbb{C}^m \) and \( \overline{\mathbb{C}}^m \) (we should emphasise that all linear maps here are assumed complex linear unless the contrary is explicitly stated).

We let \( \mathbb{C}^{m*} = L(\mathbb{C}^m, \mathbb{C}) \) denote the conjugate dual space of \( \mathbb{C}^m \). If \( A \in L(\mathbb{C}^m, \mathbb{C}^n) \), we define the transpose \( A^* \in L(\mathbb{C}^{n*}, \mathbb{C}^{m*}) \) exactly as in the real case.

We let \( \overline{\mathbb{C}}^{m*} \) denote the complex vector space of conjugate complex linear maps on \( \mathbb{C}^m \). We call \( \overline{\mathbb{C}}^{m*} \) the conjugate dual space of \( \mathbb{C}^m \). Note that \( f \in \overline{\mathbb{C}}^{m*} \) if \( f: \mathbb{C}^m \rightarrow \mathbb{C} \) is \( \mathbb{R} \)-linear and \( f(\lambda Z) = \overline{\lambda}f(Z), \lambda \in \mathbb{C}, Z \in \mathbb{C}^m \). Clearly, \( \overline{\mathbb{C}}^{m*} \approx L(\overline{\mathbb{C}}^m, \mathbb{C}) \). Moreover, \( \overline{\mathbb{C}}^{m*} \approx \overline{\mathbb{C}^{m*}} \) where we map \( \phi \in \mathbb{C}^{m*} \) to \( \overline{\phi} \in \overline{\mathbb{C}}^{m*} \) and \( \overline{\phi}(e) = \overline{\phi}(e), e \in \mathbb{C}^m \). If \( A \in L(\mathbb{C}^m, \mathbb{C}^n) \), we define the conjugate transpose \( \overline{A}^* \in L(\overline{\mathbb{C}}^{n*}, \overline{\mathbb{C}}^{m*}) \) by

\[
\overline{A}^*(\phi)(e) = \phi(A(e)), \quad e \in \mathbb{C}^m, \phi \in \overline{\mathbb{C}}^{m*}.
\]

Finally, if \( A \in L(\mathbb{C}^m, \mathbb{C}^n) \) has matrix \( [a_{ij}] \) relative to the standard bases of \( \mathbb{C}^m \) and \( \mathbb{C}^n \), the reader may easily verify that the matrices of \( \overline{A} \), \( A^* \) and \( \overline{A}^* \) are respectively \( [\overline{a}_{ij}] \), \( [a_{ji}] \) and \( [\overline{a}_{ji}] \) relative to the naturally induced bases on \( \overline{\mathbb{C}}^m \), ..., \( \overline{\mathbb{C}}^{n*} \).

Example 6. Let E be a complex vector bundle with transition functions \( \theta_{ij} : U_{ij} \rightarrow \mathrm{GL}(\mathbb{C}^m) \). We define maps
\[
\begin{aligned}
&\bar{\theta}_{ij} : U_{ij} \rightarrow \mathrm{GL}(\overline{\mathbb{C}}^m) \\
&\theta_{ij}^* : U_{ij} \rightarrow \mathrm{GL}(\mathbb{C}^{m*}), \\
&\bar{\theta}_{ij}^* : U_{ij} \rightarrow \mathrm{GL}(\overline{\mathbb{C}}^{m*})
\end{aligned}
\]
by \( \bar{\theta}_{ij}(x) = \overline{\theta_{ij}(x)}, \theta_{ij}^*(x) = [\theta_{ij}(x)^*]^{-1}, \bar{\theta}_{ij}^*(x) = [\overline{\theta_{ij}(x)^*}]^{-1}, x \in U_{ij} \).

Now \( \bar{\theta}_{ij}, \theta_{ij}^*, \bar{\theta}_{ij}^* \) satisfy the cocycle condition (A) and so are the transition functions for complex vector bundles which we call the conjugate bundle, \( \bar{E} \), dual bundle, \( E^* \), and conjugate dual bundle, \( \bar{E}^* \), of E respectively.

Exactly as for the real case we may now extend the possible fibre models of complex vector bundles to include finite combinations of direct sum, tensor product, exterior and symmetric powers of \( \mathbb{C}^m \) and its conjugate and dual spaces. All these operations extend immediately to complex vector bundles over a fixed base space and we leave formal details to the reader.
\endinput