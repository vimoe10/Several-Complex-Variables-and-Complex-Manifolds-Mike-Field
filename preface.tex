% preface.tex
\chapter*{Preface}
% \addcontentsline{toc}{chapter}{Preface}

These notes, in two parts, are intended to provide a self-contained and relatively elementary introduction to functions of several complex variables and complex manifolds. They are based on courses on complex analysis that I have given at symposia at the International Centre for Theoretical Physics, Trieste, in 1972 and 1974 and various postgraduate and seminar courses held at Warwick and Sydney. Prerequisites for the reading of Part I are minimal and, in particular, I have made no significant use of differential forms, algebraic topology, differential geometry or sheaf theory. As these notes are primarily directed towards graduate and advanced undergraduate students I have included some exercises. There are also a number of references for further reading which may serve as a suitable starting point for graduate assignments or projects. I have endeavoured to give at least one reference for any result stated but not proved in the text. For the more experienced reader, who is not a specialist in complex analysis, I have included references to related topics not directly within the scope of these notes.

My aim in these notes was to give a broad introduction to several complex variables and complex manifolds and, in particular, achieve a synthesis of the theories of compact and non-compact complex manifolds. This approach is perhaps best exemplified by the conclusion of Part II where we present Grauert's pseudoconvexivity proof of the Kodaira embedding theorem. I would hope that parts I and II together comprise a useful introduction to more advanced works on complex analysis. Notably, the books by Grauert and Remmert on Stein spaces [1] and coherent analytic sheaves (forthcoming) and that of Griffiths and Harris on the Principles of Algebraic Geometry [1].

Chapter 1 of the text is devoted to functions of one complex variable and Riemann surfaces with particular emphasis on the \(\bar{\mathfrak{d}}\)-operator and the construction of meromorphic functions with specified pole and zero sets, themes that run throughout parts I and II. The presentation is geared towards generalisations to several variables and complex manifolds and most of the results, though perhaps not the proofs, should be familiar to all readers. Section 5 of the Chapter (on vector bundles) can safely be omitted on first reading. In Chapter 2, we develop the basic theory of analytic functions of several complex variables. Amongst the results and concepts discussed are Hartog's theorem on extension of analytic functions, domains of holomorphy, holomorphic convexivity, pseudoconvexivity, Levi pseudoconvexivity and the Levi problem, the Bergman kernel function, the Cousin problems. In section 5, I have given a fairly complete treatment of boundary invariants of domains in \(\mathbb{R}^n\) with \(\mathcal{C}^2\) boundary. In part this was because of the incomplete treatment of the topic in other texts on several complex variables. In Chapter 3 we prove the Weierstrass Division and Preparation theorems and give applications to the algebraic structure of power series rings and the local structure theory of analytic sets. Here, as elsewhere in the notes, I have concentrated on the structure theory of hypersurfaces leaving the much harder general structure theory of analytic sets to the references (for example, Gunning-Rossi [1], Narasimhan [3], Whitney [1] and the forthcoming text by Grauert and Remmert on coherent analytic sheaves). The chapter concludes with a section on modules over power series rings, the reading of which may be deferred until Chapter 6 of Part II. In Chapter 4 we describe a number of basic examples of complex manifolds, both compact and non-compact, and conclude with sections on the structure theory of analytic hypersurfaces and blowing up.

Part II of these notes consists of three chapters which we now briefly describe. Chapter 5 covers calculus on complex manifolds including the construction of the \(\bar{\partial}\)-operator and the Dolbeault-Grothendieck lemma. Chapter 6 is a self-contained introduction to the theory of sheaves in complex analysis. Chapter 7 is devoted to coherence and the cohomology vanishing theorems of Cartan, Grauert and Serre. Applications include Grauert's proof of the Kodaira embedding theorem.

When I originally started these notes I had intended to include chapters on complex differential geometry and the elliptic theory of the complex Laplace-Beltrami operator applied to compact and non-compact complex manifolds. For reasons of length I eventually decided to omit these topics from Parts I and II. However, the reader will find references to chapters 8 through 12 scattered throughout the text. At some future time I hope it may be possible to complete the project with these additional chapters.

A few words of guidance to the reader of Part I: There is more than enough material in these notes for a one semester course. As we make the most substantial use of Chapter 3 in Part II, the reader may prefer to omit Chapter 3 at first reading, together with those parts of Chapter 4 on meromorphic functions and analytic sets (in particular, section 6). An alternative approach would be to read Chapter 3 (omitting section 6) and conclude with selected sections of Chapter 4 including section 6 on the structure theory of analytic hypersurfaces (this last section plays an important role in Part II).

I would like to acknowledge the great debt I owe in the preparation of these notes to many authors. I especially would like to mention the books by Grauert and Remmert on *Stein Spaces*, Gunning and Rossi on *Analytic functions of Several Complex Variables* and Hörmander on *Complex Analysis in Several Variables*. This last work has perhaps had the most decisive influence on the final form of my lecture notes.

On a more personal level, it is a great pleasure for me to express thanks to Jim Eells for interesting me in the field of complex analysis back in 1970 and for his continued help and encouragement since then. Thanks also to Tzee-Char Kuo for his advice and encouragement and to the postgraduate students here at Sydney who have been so helpful with their stimulating comments, assignments and critical questioning. Last, but by no means least, may I thank Cathy Kicinski for her beautiful job of typing the bulk of my manuscript.

Mike Field

Sydney,
September, 1981.